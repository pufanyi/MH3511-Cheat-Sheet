\documentclass[9pt,landscape]{article}

\usepackage{multicol}
\usepackage{amsmath}

\input{format.tex} % formatting template

\begin{document}

\begin{multicols}{3}

\columnseprule=0.25pt

\textbf{常用代码及公式}

$\frac{1}{\sigma^2}\sum(X_i-\mu)\sim\chi^2_n$

$\frac{1}{\sigma^2}\sum(X_i-\overline{X})\sim\chi^2_{n-1}$

$\frac{\overline{X}-\mu}{S/\sqrt{n}}\sim t_{n-1}$ ($n\ge 30$ 当 $\mathcal{N}$)

$\mathbb{P}(X>z_\alpha)=\alpha$

\texttt{dnorm} is PDF, \texttt{pnorm} is CDF, \texttt{qnorm}: quantile, $z_{1-\alpha}$, and \texttt{rnorm} 生成数据.

\textbf{Hypothesis Testing}
\begin{enumerate}
	\item rejecting the null hypothesis in favor of the alternative hypothesis
	\item there is not enough evidence to support the alternative hypothesis
\end{enumerate}

Type I error $\alpha$: reject a true $H_0$, FP

Type II error $\beta$: don't reject a false $H_0$, FN

$1-\beta$: power

\textbf{Scales of Measurement}

\begin{enumerate}
	\item nominal: 没有 order 的 categories
	\item ordinal: 有 order
	\item interval: 数值按照等长区间分类
	\item ratio: 单点的数值
\end{enumerate}

\textbf{数据的分类}
\begin{enumerate}
	\item Categorical / Qualitative: Nominal / Ordinal
	\item Numerical / Quantitative: Discrete / Continuous
\end{enumerate}

\textbf{Basic Quantities}

\texttt{quantile(arr, 0.25)}: $Q_1$

$Q_{1, 2, 3}$: $25\%$, $50\%$, $75\%$ percentile

$\mathrm{IQR}=Q_3-Q_1$

Skewness: 看尾巴在哪边
\begin{enumerate}
	\item Left-Skewed: Negative Skewness
	\item Right-Skewed: Positive Skewness
\end{enumerate}

\textbf{Why trimmed mean?}
\begin{enumerate}
	\item May have a lower SE when data is not normal
	\item Balance between median and mean, protect against outliers
\end{enumerate}

\textbf{画图}
\begin{enumerate}
	\item Stem and leaf plot: 左边是数字第一位,右边是后面的,中间用 \texttt{|} 隔开(\texttt{stem(x)})
	\item Histogram: \texttt{hist(x)}
\end{enumerate}

\textbf{transformation}

log 把中心往右,exp 把中心往左

Log-normal distribution: $\log X\sim\mathcal{N}(\mu,\sigma^2)$

$f(x)=\frac{1}{x\sigma\sqrt{2\pi}}e^{-\frac{(\log x-\mu)}{2\sigma^2}}$

$\mu=e^{\mu+\frac{\sigma^2}{2}}$, $\sigma^2=[\exp(\sigma^2)-1]\exp(2\mu+\sigma^2)$

\textbf{Coefficient of Variation (CV)}: $\frac{\sigma}{\mu}$

$\mathrm{Geomean}=\sqrt[n]{\prod X_i}$

\textbf{Imposing a Normal PDF on the Histogram}

\begin{lstlisting}
	hist(x)
	xpt <- seq(from, to, by=by)
	n_den <- dnorm(xpt, mean(return), sd(return))
	ypt <- n_den * length(x) * 10
	# We notice that each data point in the return dataset represents an area of 1 * 10, so the total area of the histogram would be * 10. 
	lines(xpt, ypt, col="blue")
\end{lstlisting}

\textbf{QQplot}: $\left(\Phi^{-1}(q_i), \hat{F}^{-1}_x(q_i)\right)$
\begin{enumerate}
	\item 左侧越低表示 longer left tail
	\item 右侧越高表示 longer right tail
\end{enumerate}

left skew 是两侧都高,right skew 是两侧都低

t 两个尾巴都长,是左低右高

\textbf{Shapiro-Wilk Test}: \texttt{shapiro.test(x)}

$W=\frac{\left(\sum_{i=1}^{n}a_ix_i\right)^2}{\sum_{i=1}^{n}(x_i-\overline{x})^2}$,$a_i$ 是这个系统自带的常数

$p$ 越小越 normal

Limitations:
\begin{enumerate}
	\item Adversely affected when there are tied data
	\item Has a bias by sample size. Statistically significant result, large sample.
\end{enumerate}

\textbf{box-plot}: \texttt{boxplot(x\textasciitilde group, data=x)}

\includegraphics[width=0.7\columnwidth]{imgs/boxplot}

\textbf{Outliers}

classic tech: $|x_i-\overline{x}|>2\cdot\mathrm{sd}$

boxplot rule: $x_i<Q_1-1.5\cdot\mathrm{IQR}$ or $x_i > Q_3+1.5\cdot\mathrm{IQR}$

\textbf{Sampling Distribution}: 那个 stat 的分布

\textbf{Confidence Interval (CI)}: $\mathbb{P}(L\le \theta\le U)=1-\alpha$

$1-\alpha$: confidence coefficient / degree of confidence

$X\sim\mathcal{N}(\mu,\sigma): \overline{X}-z_{\alpha/2}\cdot\frac{\sigma}{\sqrt{n}}<\mu<\overline{X}+z_{\alpha/2}\cdot\frac{\sigma}{\sqrt{n}}$

$X\sim\mathcal{N}(\mu,*): \overline{X}-t_{\alpha/2,n-1}\cdot\frac{S}{\sqrt{n}}<\mu<\overline{X}+t_{\alpha/2,n-1}\cdot\frac{S}{\sqrt{n}}$

\textbf{t-test}: \texttt{t.test(x, conf.level=0.95, mu=0)}

\texttt{alt="less"} 就是如果真实 $\mu$ 比较大不会拒

\texttt{alt="greater"} 就是如果真实 $\mu$ 比较小不会拒

会给 conf interval,不管 \texttt{alt} 和 \texttt{mu} 给的都是一样的

\textbf{Proportion Test}:

有可能会不合理: $\hat{p}-z_{\frac{\alpha}{2}}\sqrt{\frac{\hat{p}(1-\hat{p})}{n}}<p<\hat{p}+z_{\frac{\alpha}{2}}\sqrt{\frac{\hat{p}(1-\hat{p})}{n}}$

\texttt{prop.test(x, n, p=.5, conf.level=.95, alt)}


\end{multicols}

\end{document}
